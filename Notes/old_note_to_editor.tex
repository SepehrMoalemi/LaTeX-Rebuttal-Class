\thispagestyle{empty}
\setcounter{page}{0}
\WatermarkText{Old Note to AE}{4cm}
\begin{center}
    {\LARGE\textbf{Our Note to the Editor Persuading them to Reconsider our Submission}}
\end{center}
\begin{flushleft}
    Dear Professor Olaru,

    %%%%%%%%%%%%%%%%%%%%%%%%%%%%%%%%%%%%%%%%%%%%%%%
    On behalf of my co-author, Prof. Forbes, and I, we thank you for reviewing our submission ``Input-Output Stability of Gradient Descent: A Passivity-Based Approach'' to IEEE Control Systems Letters.

    In general, it appears our work was well-received by the majority of the reviewers, and we appreciate their valuable feedback. Both Reviewer 1 and Reviewer 7 found our paper to be ``well-written''. Moreover, Reviewer 1 agrees that our first two contributions are substantial and suggest that we remove the sections relating to our 3rd and 4th contributions as ``the paper is good without them''.
    
    As we understand, our paper was rejected based on the concerns regarding the novelty and significance of our work. We have reviewed the comments and papers provided by the reviewers. In particular, we have carefully looked at the papers cited by Reviewer 8 as papers that present our results. Bluntly, these papers do \textbf{not} present the same results as our paper.

    To elaborate further, we have looked at the works of Dörfler, Pavel, Na Li, and Notarstefano, as suggested by Reviewer 8. The passivity-based analysis of the GD method considered in these works is a \textbf{continues-time} analysis. In fact, Reviewer 8's reference to ``A Passivity-Based Approach to Nash Equilibrium Seeking Over Networks'' by Gadjov and Pavel exclusively considers the passivity of GD in \textbf{continues-time}. However, in \textbf{discrete-time}, which is the focus of our paper, GD \textbf{cannot} be interpreted as passive, without a feedthrough (D matrix) term \mbox{[15, Remark 2.7]}. Consequently, the first contribution of our paper, as found in Section~III, is to provide a passive interpretation of the GD method in discrete-time by introducing a feedthrough term via a loop transformation. As such, Reviewer 8's claim that our results are ``well known'' is false.

    Lastly, we believe Reviewer 7 misinterpreted the similarities between our work and the existing literature in [3-6]. In the introduction, we explicitly state that [3-4] use the circle criterion, [5] uses the small-gain theorem, and [6] uses a dissipativity argument to analyze the stability of first-order optimization algorithms. Contrary to [3-6], we consider a passivity-based analysis of the GD method. The benefit of this approach is that we can guarantee the input-output stability of the GD method for a \textbf{larger step size} using the weak passivity theorem. Our paper is similar to [3-6] in that we also consider a general class of sector-bounded functions. As such, Reviewer 7's claim that ``contribution (1) seems to be inappropriate, as earlier the paper says that something similar is done in [3-6]'' is, we think, a misinterpretation of what our paper claims relative to existing literature.

    Given the overall positive feedback from the reviewers and the clarification provided above regarding the novelty of our work, we are reaching out to you for a chance to reconsider our submission.
    %%%%%%%%%%%%%%%%%%%%%%%%%%%%%%%%%%%%%%%%%%%%%%%

    Thank you for your time,\\%
    \theauthor
\end{flushleft}
\newpage