% Dear Reviewers,

% We appreciate your time and effort in reviewing our work. Thanks to your constructive comments and insightful suggestions, we have made the following major changes to the manuscript:
% \begin{itemize}
%     \item {%
%         The introduction now provides a more detailed discussion of the existing gain-scheduling literature and their limitations.  
%     }%
%     \item{%
%         Section~III has been heavily revised to introduce the matrix-gain-scheduling architecture. We further discuss in detail the objective of the paper and introduce the required scheduling matrix properties. Finally, we provide a detailed discussion on how to design the scheduling matrices to satisfy such properties. We show that the proposed matrix-gain-scheduling architecture allows for greater freedom in the design of the scheduling functions compared to the gain-scheduling architecture in \cite{moalemi_forbes_ccta,QSR}. 
%     }%
%     \item{%
%         Figures 1, 2, 3, and 5 are modified to better convey the matrix-vector multiplication between the scheduling matrices and their corresponding input signals. When appropriate, we have added additional details to better portray that $N$ subsystems are being considered. 
%     }%
%     \item{%
%         Removed the scaling terms $\alpha_i$ from the gain-scheduling architecture to keep the focus on the scheduling matrices and reduce complexity. Since we used \(\alpha_i = 1\) when presenting the simulation results in the original manuscript, the results in the revised manuscript remain unchanged.
%     }%
%     \item{%
%         The scheduling matrices are now solely a function of time. Additionally, rather than considering \(t \in [0, T]\), all \(\sup\) and \(\inf\) operations are performed over \(t \in \mathbb{R}_{\geq0}\). This removes the dependency of the proposed results on \(T\).
%     }%
% \end{itemize}
% Below, we have addressed your specific comments and suggestions. We have also appended the revised manuscript with the major changes highlighted in \textcolor{blue}{blue}. 

% We believe that our paper is now more accessible to readers, the contributions are more clearly presented, and the results are more thorough. We thank you for helping us improve the quality of our work.

% Sincerely,\\[1em]%
% \theauthor
\vspace{-15pt}
\section*{Report}
Three Reviewers with diverse expertise have assessed the manuscript.

Reviewer 1 observes that
\vspace{-7pt}
\begin{itemize}[nosep]
    \item {%
        the contributions are building on standard results
    }%
    \item{%
        of the four claimed contributions, only the first two are actually reported in the manuscript with sufficient detail
    }%
    \item{%
        as for the third contribution, the approach is taken from existing literature and does not appear rigorous
    }%
    \item{%
        both the third and the fourth contribution are explained with insufficient detail and thus unclear.
    }%
\end{itemize}

Reviewer 7 highlights that
\vspace{-7pt}
\begin{itemize}[nosep]
    \item {%
        claiming the first contribution is inappropriate, because something similar has already been done in the past
    }%
    \item{%
        the main contribution is in any case minor (a small modification / reformulation) and its benefits are unclear
    }%
    \item{%
        the manuscript does not offer any insight that was not known before through other methods
    }%
    \item{%
        the third contribution raises concerns and the benefits of the presented approach appear to be minimal
    }%
\end{itemize}


Reviewer 8 points out that, although the work is mostly technically correct, apart from e.g.\ a missing technical assumption on the boundedness of the sub-differential of the objective function,
\vspace{-7pt}
\begin{itemize}[nosep]
    \item {%
        a vast literature has been neglected (see the works of Notarstefano, Na Li, Dorfler, Pavel, etc)
    }%
    \item{%
        the presented results are well known, even in more general terms (see D. Gadjov and L. Pavel, ``A Passivity-Based Approach to Nash Equilibrium Seeking Over Networks,'' in IEEE Transactions on Automatic Control, vol. 64, no. 3, pp. 1077-1092, March 2019, doi: 10.1109/TAC.2018.2833140.)
    }%
    \item{%
        the conclusions in the numerical results section are therefore to be expected
    }%
    \item{%
        the control theoretic and system theoretic interpretation to the gradient descent algorithm in convex optimization have already been proposed in previous literature
    }%
    \item{%
        all in all, no novel theoretical, practical or technical contribution can be found in the manuscript
    }%
\end{itemize}

In light of the serious concerns expressed by the Reviewers, the novelty and significance of the contribution appear insufficient for a journal paper. They might be considered as a potential conference contribution for ACC, at the discretion of the program committee.