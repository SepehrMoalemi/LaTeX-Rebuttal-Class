\section*{Reviewer 8}\label{sec:reviewer8}
\renewcommand{\theequation}{R8.\arabic{equation}}
\setcounter{equation}{0}
% -------------------------------------------------------------------------------------------- %
\begin{rebuttal}[resovled]
    % Comment
    {%
        It is well known that the gradient descent dynamics for unconstrained (convex) optimization problems are passive, even more generally in distributed optimization and multi-agent equilibrium problems, and even subject to convex constraints.
    }%
    % Response
    {%
    }%
\end{rebuttal}
% -------------------------------------------------------------------------------------------- %
\begin{rebuttal}[resovled]
    % Comment
    {%
        It is also known that asymptotic convergence to an optimal solution holds if the step size are taken small enough. There are several works in the literature that quantify how small the step size should be and how fast a time-varying step size should vanish. The conclusions made in Section V, e.g.\ summarized in Table I,  are therefore to be expected.
    }%
    % Response
    {%
    }%
\end{rebuttal}
% -------------------------------------------------------------------------------------------- %
\begin{rebuttal}[resovled]
    % Comment
    {%
    From this perspective, both the ``Control interpretation'' in Section II  and the convergence results (Theorem 3) in Section III are - mathematically speaking - already known. Furthermore, there are already several papers that attach a system theoretic interpretation to the gradient descent algorithm in convex optimization.
    }%
    % Response
    {%
    }%
\end{rebuttal}
% -------------------------------------------------------------------------------------------- %
\begin{rebuttal}[resovled]
    % Comment
    {%
    Consequently, this reviewer is not clear neither on the novel theoretical contribution of the paper, nor on the more practical, technical contributions. In turn, this reviewer cannot see the implications of the paper for the field of operations research and/or for the field of systems and control theory.
    }%
    % Response
    {%
    }%
\end{rebuttal}
% -------------------------------------------------------------------------------------------- %
\begin{rebuttal}[stuck]
    % Comment
    {%
    From a technical perspective, given that potential non-convexity of the $f$ function, it is unclear what is $\mbf{x}^{\ast}$ in Section IV\@. Furthermore, one should assume that the sub-differential of $f$ is bounded, a mathematical detail that is omitted in the paper.
    }%
    % Response
    {%
    }%
\end{rebuttal}
% -------------------------------------------------------------------------------------------- %