\thispagestyle{empty}
\setcounter{page}{0}
% ! Somewhere talk about CCTA being accepted
% ! Recommend other reviewers
% ! Maybe give credit to reviewer 10 for pointing out the time dependency on T?
% ! fix the verb tenses
\begin{flushleft}
    Dear Associate Editor,

    %%%%%%%%%%%%%%%%%%%%%%%%%%%%%%%%%%%%%%%%%%%%%%%
    Thank you for the opportunity to re-submit our work to the IEEE Transactions on Automatic Control. We have incorporated the reviewers' helpful comments into our revised submission, and address each comment individually in the attached rebuttal document. In general, we believe our work was well-received by the reviewers. 
    
    Reviewer 2 agrees that our paper is a ``solid contribution'' and provides a thorough presentation of our theoretical contributions. 
    
    Reviewer 7 mainly focused on the control synthesis aspect of the simulation and also provided helpful suggestions for improving the clarity of our paper.
    
    Reviewer 10 appears to be less familiar with gain-scheduling and the role of the scheduling functions as design variables. We would like to discuss the following comments provided by this reviewer:
    \begin{quote}\enquote{\textit{%
        However, sometimes the authors claim to choose or design specific scheduling matrices which is very confusing.
        }}%
    \end{quote}
    \begin{quote}\enquote{\textit{%
        Also, in subsection A: scheduling matrix construction, the authors mention a few times design of the scheduling matrices.%
        }}%
    \end{quote}
    It is clear that Reviewer 10 did not realize that the scheduling matrices are free design variables. Based on this fundamental misunderstanding, the reviewer also states
    \begin{quote}\enquote{\textit{%
        Obviously, this [pseudo-commutativity] prerequisite is too strong and restrictive which greatly blocks the applicability of the presented method to most systems. Especially, this pseudo-commutativity is with respect to the dissipativity matrices \(\mbf{S}_i\) which is hard to satisfy in general.%
        }}%
    \end{quote}
    However, in Section~IV-A of our initial submission, we explicitly show that the scheduling matrices can be constructed to satisfy the pseudo-commutativity condition for any choice of \(\mbf{S}_i\). In the same section, we also show that the existing matrix-gain-scheduling architecture is a special case of our proposed architecture. 

    Finally, the specific comments provided by Reviewer 10 cover our paper up to Theorem~1 on page~4. The reviewer then states that ``similar issues and/or flaws exist for the rest of the paper'' without providing further detail. To the best of our ability, we have addressed the comments provided by Reviewer 10 and applied the changes to the rest of the paper.
    %%%%%%%%%%%%%%%%%%%%%%%%%%%%%%%%%%%%%%%%%%%%%%%

    Thank you,\\[1em]%
    \theauthor
\end{flushleft}
\newpage