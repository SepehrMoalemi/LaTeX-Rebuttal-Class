\thispagestyle{empty}
\setcounter{page}{0}
\begin{flushleft}
    Dear Associate Editor,

    %%%%%%%%%%%%%%%%%%%%%%%%%%%%%%%%%%%%%%%%%%%%%%%
    Thank you for reviewing our submission to IEEE Control Systems Letters.

    In general, it appears our initial work was well-received by the majority of the reviewers, and we appreciate their valuable feedback. Both Reviewer 1 and Reviewer 7 found our paper to be ``well-written''. In fact, Reviewer 1 agrees that our first two contributions are substantial and suggest that we remove the sections relating to our 3rd and 4th contributions as ``the paper is good without them''.
    
    As we understand, our paper was rejected based on the concerns regarding the novelty and significance of our work. We have reviewed the comments provided by the reviewers and would like to address the comments regarding the novelty of our work.

    As mentioned by Reviewer 8, the literature on this topic is vast. Since our paper exclusively considers the discrete-time passivity-based analysis of the gradient descent (GD) method, we provided a brief overview of other discrete-time input-output stability results in [3-10]. We appreciate the references provided by Reviewer 8 and have looked at the works of Dorfler, Pavel, Na Li, and Notarstefano. The passivity-based analysis of the GD method considered in these works is based in continuous-time. In fact, the reviewer's reference to ``A Passivity-Based Approach to Nash Equilibrium Seeking Over Networks'' by Gadjov and Pavel exclusively considers the passivity of GD in continues-time. However, in discrete time, the existing control interpretation of the GD method is strictly proper and therefore, can never be passive [15, Remark 2.7]. Consequently, the main contribution of our paper, as found in Section~III, is to provide a passive interpretation of the GD method in discrete-time.

    Lastly, Reviewer 7 mentions that ``in the introduction, contribution (1) seems to be inappropriate, as earlier the paper says that something similar is done in [3-6]''. However, in the introduction, we explicitly state that [3-4] use the circle criterion, [5] uses the small-gain theorem, and [6] uses a dissipativity argument. Contrary to [3-6], we consider a passivity-based analysis of the GD method. The benefit of this approach is that we can guarantee the input-output stability of the GD method for a larger step size using the weak passivity theorem. Our paper is similar to [3-6] in that we also consider a general class of sector-bounded functions. We believe this is a significant novel contribution.

    \textcolor{blue}{Something here asking them to reconsider the decision based on the above points.}
    %%%%%%%%%%%%%%%%%%%%%%%%%%%%%%%%%%%%%%%%%%%%%%%

    Thank you,\\[1em]%
    \theauthor
\end{flushleft}
\newpage